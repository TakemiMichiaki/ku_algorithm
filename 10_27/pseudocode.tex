%定義

%\documentclass[10pt,twocolumn]{jarticle} 
\documentclass[10pt]{jarticle} 
\usepackage{graphicx}
\usepackage{fancybox}
\usepackage{comment}
\usepackage{amsmath}
\usepackage{amssymb}
\usepackage{amsfonts}
\usepackage{euler}
\usepackage{color}
\usepackage{algorithm}
%\usepackage{algorithmic}
\usepackage{algpseudocode}
\usepackage{pseudocode}
\algtext*{EndWhile}
\algtext*{EndIf}
\algtext*{EndFor}

\newcommand\NoDo{\renewcommand\algorithmicdo{}}
\newcommand\ReDo{\renewcommand\algorithmicdo{\textbf{do}}}
\newcommand\NoThen{\renewcommand\algorithmicthen{}}
\newcommand\ReThen{\renewcommand\algorithmicthen{\textbf{then}}}


\pagestyle{empty}

%余白とか

\setlength{\topmargin}{-2cm} 
\setlength{\textheight}{26.5cm} 
\setlength{\textwidth}{18.5cm}
\setlength{\oddsidemargin}{-1.3cm} 
\setlength{\columnsep}{.5cm}
\newcommand{\noin}{\noindent}
\catcode`@=\active \def@{\hspace{0.9bp}-\hspace{0.9bp}}
\newtheorem{dfn}{定義}[section]


%タイトル

\title{第二回 アルゴリズム勉強会}
\setcounter{footnote}{1}
%\author{原田 崇司\if0\thanks{神奈川大学大学院理学研究科情報科学専攻 田中研究室}\fi}
\author{原田 崇司\thanks{神奈川大学大学院 理学研究科 情報科学専攻 田中研究室}}
\date{\today}
\西暦

%タイトル作成

\begin{document}

\maketitle
\thispagestyle{empty}

% 1. 何について話すか(概要) => 2. 具体例(図だけで十分) => 3. 結論(一番伝えたい事)
%
% 結論に向けて, 流れを意識して書く.
%
% 構文. 主語, 述語, 目的語をはっきり書く.
%
% 受動態を使わない.
%
% 辞書, 参考書, 先生に伺うことより, 適切な単語, 表現で文章を作成する.
%
% 内容を削る. 必要最小限
%
% コンマを列挙以外で2つ以上使う, ``~の~の'', の如きは使わない.
\noindent{\Large {\color{red} 一流の科学者でも間違えます.原田の言うことは間違っていると思って参加して下さい.}}

%\section{The Role of Algorithms in Computing}

\begin{algorithm}
% \caption{INSERTION-SORT($A$)}
 INSERTION-SORT($A$)
 \begin{algorithmic}[1]
  \NoDo
  \For{$j = 2$}\TO{$A.length$}
   \State $key = A[j]$
   \State // Insert $A[j]$ into the sorted sequence $A[1 \, .. \, j-1]$.
   \State $i = j - 1$
   \While{$i > 0$ and $A[i] > key$}
    \State $A[i+1] = A[i]$
    \State $i = i - 1$
   \EndWhile
   \State $A[i+1] = key$
  \EndFor
 \end{algorithmic}
\end{algorithm}

\begin{algorithm}
 LINEAR-SEARCH($A, v$)
 \begin{algorithmic}[1]
  \State $i =$ NIL
  \NoDo
  \For{$j = 1$}\TO{$A.length$}
   \NoThen
   \If{$A[j] == v$}
    \State $i = j$
    \State \RETURN{$i$}
   \EndIf
  \EndFor
  \State \RETURN{NIL}
 \end{algorithmic}
\end{algorithm}

%% \begin{pseudocode}{linear-search}{\empty}
%%   \FOR i = 0 \TO 10 \\
%%     i += 1
%% \end{pseudocode}


\end{document}
