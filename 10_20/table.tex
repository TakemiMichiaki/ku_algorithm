%定義

%\documentclass[10pt,twocolumn]{jarticle} 
\documentclass[10pt]{jarticle} 
\usepackage{graphicx}
\usepackage{fancybox}
\usepackage{comment}
\usepackage{amsmath}
\usepackage{amssymb}
\usepackage{amsfonts}
\usepackage{euler}
\usepackage{color}

\pagestyle{empty}

%余白とか

\setlength{\topmargin}{-2cm} 
\setlength{\textheight}{26.5cm} 
\setlength{\textwidth}{18.5cm}
\setlength{\oddsidemargin}{-1.3cm} 
\setlength{\columnsep}{.5cm}
\newcommand{\noin}{\noindent}
\catcode`@=\active \def@{\hspace{0.9bp}-\hspace{0.9bp}}
\newtheorem{dfn}{定義}[section]


%タイトル

\title{第一回 アルゴリズム勉強会}
\setcounter{footnote}{1}
%\author{原田 崇司\if0\thanks{神奈川大学大学院理学研究科情報科学専攻 田中研究室}\fi}
\author{原田 崇司\thanks{神奈川大学大学院 理学研究科 情報科学専攻 田中研究室}}
\date{\today}
\西暦

%タイトル作成

\begin{document}

\maketitle
\thispagestyle{empty}

% 1. 何について話すか(概要) => 2. 具体例(図だけで十分) => 3. 結論(一番伝えたい事)
%
% 結論に向けて, 流れを意識して書く.
%
% 構文. 主語, 述語, 目的語をはっきり書く.
%
% 受動態を使わない.
%
% 辞書, 参考書, 先生に伺うことより, 適切な単語, 表現で文章を作成する.
%
% 内容を削る. 必要最小限
%
% コンマを列挙以外で2つ以上使う, ``~の~の'', の如きは使わない.
\noindent{\Large {\color{red} 一流の科学者でも間違えます.原田の言うことは間違っていると思って参加して下さい.}}

\section{The Role of Algorithms in Computing}
\begin{itemize}
 \item アルゴリズムとは何か 
  \begin{itemize}
   \item 適当に定義された(解釈の仕方が一通りしかない)計算手続き
   \item ある入力から,それに対応する出力へと変換する計算ステップ
   \item 適当に記述された計算問題(computational problem)を解くツール
   \item 有限性(Finiteness),明確性(Difiniteness)、入力(Input),出力(Output),実効性(Effectiveness)の五つの特徴を持つ演算(operation)のシーケンス(Knuth の定義.というより,多分,一般のアルゴリズムの定義)
    \begin{itemize}
     \item 有限性(Finiteness): アルゴリズムは,有限のステップで停止しなければならない.
     \item 明確性(Difiniteness): アルゴリズムの各ステップは,正確に定義されてなければならない.例えば,アルゴリズムのステップに,''$10 \div 0$ の商'' などの定義されていない演算を用いることはできない.
     \item 入力(Input): アルゴリズムは,$0$以上の入力を持つ.
     \item 出力(Output): アルゴリズムは,$1$以上の出力を持つ.
     \item 実効性(Effectiveness): アルゴリズムは,紙と鉛筆を用いて実際に確かめることができなければならない.例えば,十進展開の無限列や,物理的に紙に書かれた線分の長さなどを,演算の対象とするとことはできない.また,''$P = NP$ ならば,$n = n + 1$ を行う'' などの,未だに解けてない問題の解を利用する演算は許さない.
    \end{itemize}
  \end{itemize}
 \item どうしてアルゴリズムが必要なのか
\end{itemize}
\begin{table}[ht]
 \centering{
  %\caption{ルールリスト}
  \begin{tabular}{c|c|c|c|c|c|c|c}
      & $1$ second & $1$ minute & $1$ hour & $1$ day & $1$ month & $1$ year & $1$ century \\ \hline
$\lg n$ &          &            &          &         &           &          & \\ \hline
$\sqrt{n}$ &          &            &          &         &           &          & \\ \hline
$n$ & $10^{6}$ & $6 \times 10^{7}$ & $3.6 \times 10^{9}$ & $8.64 \times 10^{10}$ & $2.59 \times 10^{12}$ & $9.46 \times 10^{14}$ & $9.46 \times 10^{16}$ \\ \hline
$n \lg n$     &          &            &          &         &           &          & \\ \hline
$n^{2}$     &          &            &          &         &           &          & \\ \hline
$n^{3}$     &          &            &          &         &           &          & \\ \hline
$2^{n}$     &          &            &          &         &           &          & \\ \hline
$n!$     &          &            &          &         &           &          & 
  \end{tabular}
  %\label{growth table}
}
\end{table}

\end{document}
